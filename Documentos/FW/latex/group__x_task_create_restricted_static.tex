\hypertarget{group__x_task_create_restricted_static}{}\section{x\+Task\+Create\+Restricted\+Static}
\label{group__x_task_create_restricted_static}\index{x\+Task\+Create\+Restricted\+Static@{x\+Task\+Create\+Restricted\+Static}}
task. h 
\begin{DoxyPre}
 BaseType\_t xTaskCreateRestrictedStatic( TaskParameters\_t *pxTaskDefinition, TaskHandle\_t *pxCreatedTask );\end{DoxyPre}


Only available when config\+S\+U\+P\+P\+O\+R\+T\+\_\+\+S\+T\+A\+T\+I\+C\+\_\+\+A\+L\+L\+O\+C\+A\+T\+I\+ON is set to 1.

x\+Task\+Create\+Restricted\+Static() should only be used in systems that include an M\+PU implementation.

Internally, within the Free\+R\+T\+OS implementation, tasks use two blocks of memory. The first block is used to hold the task\textquotesingle{}s data structures. The second block is used by the task as its stack. If a task is created using x\+Task\+Create\+Restricted() then the stack is provided by the application writer, and the memory used to hold the task\textquotesingle{}s data structure is automatically dynamically allocated inside the x\+Task\+Create\+Restricted() function. If a task is created using x\+Task\+Create\+Restricted\+Static() then the application writer must provide the memory used to hold the task\textquotesingle{}s data structures too. x\+Task\+Create\+Restricted\+Static() therefore allows a memory protected task to be created without using any dynamic memory allocation.


\begin{DoxyParams}{Parameters}
{\em px\+Task\+Definition} & Pointer to a structure that contains a member for each of the normal x\+Task\+Create() parameters (see the x\+Task\+Create() A\+PI documentation) plus an optional stack buffer and the memory region definitions. If config\+S\+U\+P\+P\+O\+R\+T\+\_\+\+S\+T\+A\+T\+I\+C\+\_\+\+A\+L\+L\+O\+C\+A\+T\+I\+ON is set to 1 the structure contains an additional member, which is used to point to a variable of type Static\+Task\+\_\+t -\/ which is then used to hold the task\textquotesingle{}s data structure.\\
\hline
{\em px\+Created\+Task} & Used to pass back a handle by which the created task can be referenced.\\
\hline
\end{DoxyParams}
\begin{DoxyReturn}{Returns}
pd\+P\+A\+SS if the task was successfully created and added to a ready list, otherwise an error code defined in the file \hyperlink{projdefs_8h}{projdefs.\+h}
\end{DoxyReturn}
Example usage\+: 
\begin{DoxyPre}
// Create an TaskParameters\_t structure that defines the task to be created.
// The StaticTask\_t variable is only included in the structure when
// configSUPPORT\_STATIC\_ALLOCATION is set to 1.  The PRIVILEGED\_DATA macro can
// be used to force the variable into the RTOS kernel's privileged data area.
static PRIVILEGED\_DATA StaticTask\_t xTaskBuffer;
static const TaskParameters\_t xCheckTaskParameters =
\{
    vATask,     // pvTaskCode - the function that implements the task.
    "ATask",    // pcName - just a text name for the task to assist debugging.
    100,        // usStackDepth - the stack size DEFINED IN WORDS.
    NULL,       // pvParameters - passed into the task function as the function parameters.
    ( 1UL | portPRIVILEGE\_BIT ),// uxPriority - task priority, set the portPRIVILEGE\_BIT if the task should run in a privileged state.
    cStackBuffer,// puxStackBuffer - the buffer to be used as the task stack.\end{DoxyPre}



\begin{DoxyPre}    // xRegions - Allocate up to three separate memory regions for access by
    // the task, with appropriate access permissions.  Different processors have
    // different memory alignment requirements - refer to the FreeRTOS documentation
    // for full information.
    \{
        // Base address                 Length  Parameters
        \{ cReadWriteArray,              32,     portMPU\_REGION\_READ\_WRITE \},
        \{ cReadOnlyArray,               32,     portMPU\_REGION\_READ\_ONLY \},
        \{ cPrivilegedOnlyAccessArray,   128,    portMPU\_REGION\_PRIVILEGED\_READ\_WRITE \}
    \}\end{DoxyPre}



\begin{DoxyPre}     // Holds the task's data structure.
\};\end{DoxyPre}



\begin{DoxyPre}int \hyperlink{group__main__module_ga840291bc02cba5474a4cb46a9b9566fe}{main( void )}
\{
TaskHandle\_t xHandle;
\begin{DoxyVerb}// Create a task from the const structure defined above.  The task handle
// is requested (the second parameter is not NULL) but in this case just for
// demonstration purposes as its not actually used.
xTaskCreateRestricted( &xRegTest1Parameters, &xHandle );

// Start the scheduler.
vTaskStartScheduler();

// Will only get here if there was insufficient memory to create the idle
// and/or timer task.
for( ;; );
\end{DoxyVerb}

\}
   \end{DoxyPre}
 