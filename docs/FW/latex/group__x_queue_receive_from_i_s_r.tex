\hypertarget{group__x_queue_receive_from_i_s_r}{}\section{x\+Queue\+Receive\+From\+I\+SR}
\label{group__x_queue_receive_from_i_s_r}\index{x\+Queue\+Receive\+From\+I\+SR@{x\+Queue\+Receive\+From\+I\+SR}}
queue. h 
\begin{DoxyPre}
BaseType\_t xQueueReceiveFromISR(
                                   QueueHandle\_t    xQueue,
                                   void *pvBuffer,
                                   BaseType\_t *pxTaskWoken
                               );
  \end{DoxyPre}


Receive an item from a queue. It is safe to use this function from within an interrupt service routine.


\begin{DoxyParams}{Parameters}
{\em x\+Queue} & The handle to the queue from which the item is to be received.\\
\hline
{\em pv\+Buffer} & Pointer to the buffer into which the received item will be copied.\\
\hline
{\em px\+Task\+Woken} & A task may be blocked waiting for space to become available on the queue. If x\+Queue\+Receive\+From\+I\+SR causes such a task to unblock $\ast$px\+Task\+Woken will get set to pd\+T\+R\+UE, otherwise $\ast$px\+Task\+Woken will remain unchanged.\\
\hline
\end{DoxyParams}
\begin{DoxyReturn}{Returns}
pd\+T\+R\+UE if an item was successfully received from the queue, otherwise pd\+F\+A\+L\+SE.
\end{DoxyReturn}
Example usage\+: 
\begin{DoxyPre}\end{DoxyPre}



\begin{DoxyPre}QueueHandle\_t xQueue;\end{DoxyPre}



\begin{DoxyPre}// Function to create a queue and post some values.
void vAFunction( void *pvParameters )
\{
char cValueToPost;
const TickType\_t xTicksToWait = ( TickType\_t )0xff;\end{DoxyPre}



\begin{DoxyPre}   // Create a queue capable of containing 10 characters.
   xQueue = xQueueCreate( 10, sizeof( char ) );
   if( xQueue == 0 )
   \{
    // Failed to create the queue.
   \}\end{DoxyPre}



\begin{DoxyPre}   // ...\end{DoxyPre}



\begin{DoxyPre}   // Post some characters that will be used within an ISR.  If the queue
   // is full then this task will block for xTicksToWait ticks.
   cValueToPost = 'a';
   \hyperlink{queue_8h_af7eb49d3249351176992950d9185abe9}{xQueueSend( xQueue, ( void * ) &cValueToPost, xTicksToWait )};
   cValueToPost = 'b';
   \hyperlink{queue_8h_af7eb49d3249351176992950d9185abe9}{xQueueSend( xQueue, ( void * ) &cValueToPost, xTicksToWait )};\end{DoxyPre}



\begin{DoxyPre}   // ... keep posting characters ... this task may block when the queue
   // becomes full.\end{DoxyPre}



\begin{DoxyPre}   cValueToPost = 'c';
   \hyperlink{queue_8h_af7eb49d3249351176992950d9185abe9}{xQueueSend( xQueue, ( void * ) &cValueToPost, xTicksToWait )};
\}\end{DoxyPre}



\begin{DoxyPre}// ISR that outputs all the characters received on the queue.
void vISR\_Routine( void )
\{
BaseType\_t xTaskWokenByReceive = pdFALSE;
char cRxedChar;\end{DoxyPre}



\begin{DoxyPre}   while( xQueueReceiveFromISR( xQueue, ( void * ) &cRxedChar, &xTaskWokenByReceive) )
   \{
    // A character was received.  Output the character now.
    vOutputCharacter( cRxedChar );\end{DoxyPre}



\begin{DoxyPre}    // If removing the character from the queue woke the task that was
    // posting onto the queue cTaskWokenByReceive will have been set to
    // pdTRUE.  No matter how many times this loop iterates only one
    // task will be woken.
   \}\end{DoxyPre}



\begin{DoxyPre}   if( cTaskWokenByPost != ( char ) pdFALSE;
   \{
    taskYIELD ();
   \}
\}
\end{DoxyPre}
 